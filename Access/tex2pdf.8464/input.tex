\documentclass[]{article}
\usepackage[T1]{fontenc}
\usepackage{lmodern}
\usepackage{amssymb,amsmath}
\usepackage{ifxetex,ifluatex}
\usepackage{fixltx2e} % provides \textsubscript
% Set line spacing
% use upquote if available, for straight quotes in verbatim environments
\IfFileExists{upquote.sty}{\usepackage{upquote}}{}
\ifnum 0\ifxetex 1\fi\ifluatex 1\fi=0 % if pdftex
  \usepackage[utf8]{inputenc}
\else % if luatex or xelatex
  \ifxetex
    \usepackage{mathspec}
    \usepackage{xltxtra,xunicode}
  \else
    \usepackage{fontspec}
  \fi
  \defaultfontfeatures{Mapping=tex-text,Scale=MatchLowercase}
  \newcommand{\euro}{€}
\fi
% use microtype if available
\IfFileExists{microtype.sty}{\usepackage{microtype}}{}
\usepackage[margin=1in]{geometry}
\usepackage{longtable,booktabs}
\usepackage{graphicx}
% Redefine \includegraphics so that, unless explicit options are
% given, the image width will not exceed the width of the page.
% Images get their normal width if they fit onto the page, but
% are scaled down if they would overflow the margins.
\makeatletter
\def\ScaleIfNeeded{%
  \ifdim\Gin@nat@width>\linewidth
    \linewidth
  \else
    \Gin@nat@width
  \fi
}
\makeatother
\let\Oldincludegraphics\includegraphics
{%
 \catcode`\@=11\relax%
 \gdef\includegraphics{\@ifnextchar[{\Oldincludegraphics}{\Oldincludegraphics[width=\ScaleIfNeeded]}}%
}%
\ifxetex
  \usepackage[setpagesize=false, % page size defined by xetex
              unicode=false, % unicode breaks when used with xetex
              xetex]{hyperref}
\else
  \usepackage[unicode=true]{hyperref}
\fi
\hypersetup{breaklinks=true,
            bookmarks=true,
            pdfauthor={Bryan},
            pdftitle={cyanoMonDocumentation},
            colorlinks=true,
            citecolor=blue,
            urlcolor=blue,
            linkcolor=magenta,
            pdfborder={0 0 0}}
\urlstyle{same}  % don't use monospace font for urls
\setlength{\parindent}{0pt}
\setlength{\parskip}{6pt plus 2pt minus 1pt}
\setlength{\emergencystretch}{3em}  % prevent overfull lines
\setcounter{secnumdepth}{0}

%%% Change title format to be more compact
\usepackage{titling}
\setlength{\droptitle}{-2em}
  \title{cyanoMonDocumentation}
  \pretitle{\vspace{\droptitle}\centering\huge}
  \posttitle{\par}
  \author{Bryan}
  \preauthor{\centering\large\emph}
  \postauthor{\par}
  \predate{\centering\large\emph}
  \postdate{\par}
  \date{April 23, 2015}




\begin{document}

\maketitle


\subsection{To Do List}\label{to-do-list}

\begin{itemize}
\itemsep1pt\parskip0pt\parsep0pt
\item
  Build forms
\item
  Complete 2014 data QA/QC
\item
  Interface for Phone App
\item
  Update for 2015
\end{itemize}

\subsection{Considerations for 2015
database}\label{considerations-for-2015-database}

\begin{itemize}
\itemsep1pt\parskip0pt\parsep0pt
\item
  Remove Field ``analysisTime'' from table tblAnalysis
\item
  Remove Field ``rep'' from table tblFluorometry
\item
  Combine tables tblAnalysis \& tblFluorometry
\item
  Do we want to include anytime of replicate analyses or readings?
\item
  Add text field `stationDescription' to describe location of the
  station (e.g., off Bubby's dock)
\item
  JSON to google docs link
\item
  use phone app for field data only?
\item
  add tables for lab results and ancillary data-toxins, nutrients,
  secchi, etc.
\end{itemize}

\subsection{Background}\label{background}

\begin{itemize}
\itemsep1pt\parskip0pt\parsep0pt
\item
  EPA region 1 is coordinating a Cyanobacteria Monitoring Progam for the
  six New England States (CT, MA, ME, NH, RI, \& VT)
\item
  Data collection initiated during the summer of 2014
\item
  2014 data have been collated and standardized
\item
  For future data collection we need a relational database developed for
  data entry and archiving
\item
  The Database needs to work with a Data Collection Phone App under
  development
\item
  For now, the database will be created in MSAccess (cyanoMon.mdb)
\end{itemize}

\subsection{Database Structure}\label{database-structure}

\begin{itemize}
\itemsep1pt\parskip0pt\parsep0pt
\item
  The five tables in the access database (cyanoMon.mdb) are described
  below:
\end{itemize}

\includegraphics{cyanoMonRelationships.jpg}

\begin{itemize}
\itemsep1pt\parskip0pt\parsep0pt
\item
  \textbf{tblWaterbody} provides general information on the waterbody
  and assigns a unique identifier. Ideally we will have this table
  populated before the field crews go out so that they can select the
  correct lake from a list. The reality is that we will also need to be
  able to add lakes on the fly as new lakes are added to the sampling
  plan. There may be multiple stations for each waterbody.
\end{itemize}

\begin{longtable}[c]{@{}lll@{}}
\toprule\addlinespace
Field & Data Type & Description
\\\addlinespace
\midrule\endhead
\textbf{waterbodyID} & Short Text & Primary Key for this table. Unique
ID for the Waterbody. Can either be entered by the users or will be
added later.
\\\addlinespace
\textbf{waterbodyName} & Short Text & Name of the waterbody
\\\addlinespace
\textbf{state} & Short Text & Combo Box (``CT''; ``MA''; ``ME''; ``NH'';
``RI''; ``VT''): Two letter state abbreviations
\\\addlinespace
\textbf{town} & Short Text & Text Box: Closest town to the lake
\\\addlinespace
\textbf{WBID} & Long Integer & Text Box: EPA Waterbody Identifier
\\\addlinespace
\textbf{otherWaterbodyID} & Short Text & Text Box: If the states or the
samping organization have a unique identifier for the waterbody it can
be added here.
\\\addlinespace
\textbf{longitudeWB} & Double & Text Box: longitude in decimal degrees
(WGS84) of the lake centroid. This field will be populated by the
database administrator.
\\\addlinespace
\textbf{latitudeWB} & Double & Text Box: latitude in decimal degrees
(WGS84) of the lake centroid. This field will be populated by the
database administrator.
\\\addlinespace
\textbf{locationSourceWB} & Short Text & Combo Box
(``WaterbodyDatabase''; ``GPS''; ``GoogleEarth''; ``BingMaps'';
``topoMap''): How was the location determined? This field will be
populated by the database administrator.
\\\addlinespace
\textbf{commentWB} & Long Text & Text Box: Additional information or
comments
\\\addlinespace
\bottomrule
\end{longtable}

\begin{itemize}
\itemsep1pt\parskip0pt\parsep0pt
\item
  \textbf{tblStation} within each Waterbody there may be multiple
  stations. This table provides general information on the station.
  There may be multiple samples taken from each station.
\end{itemize}

\begin{longtable}[c]{@{}lll@{}}
\toprule\addlinespace
Field & Data Type & Description
\\\addlinespace
\midrule\endhead
\textbf{stationID} & Short Text & Primary Key for this table. Unique ID
for the Station
\\\addlinespace
\textbf{waterbodyID} & Short Text & Lookup primary Key from tblWaterbody
\\\addlinespace
\textbf{otherStationID} & Short Text & Text Box: If the states or the
samping organization have a unique identifier for the station it can be
added here.
\\\addlinespace
\textbf{stationType} & Short Text & List Box/Radio Button
(``nearShore''; ``offShore'';``other''): Location of the station in
relation to the shore;for special situations choose other and add notes
in field ``commentStation''
\\\addlinespace
\textbf{longitudeSta} & Double & Text Box: longitude in decimal degrees
(WGS84) of the station. Miniumum of 4 decimal places; 6 decimal places
prefered.
\\\addlinespace
\textbf{latitudeSta} & Double & Text Box: latitude in decimal degrees
(WGS84) of the station. Miniumum of 4 decimal places; 6 decimal places
prefered.
\\\addlinespace
\textbf{locationSourceSta} & Short Text & Combo Box
(``WaterbodyDatabase''; ``GPS''; ``GoogleEarth''; ``BingMaps'';
``topoMap''): How was the location determined?
\\\addlinespace
\textbf{commentStation} & Long Text & Text Box: Additional information
or comments
\\\addlinespace
\bottomrule
\end{longtable}

\begin{itemize}
\itemsep1pt\parskip0pt\parsep0pt
\item
  \textbf{tblSample} for each station within a waterbody there may be
  multiple sample events. This table provides general information on
  each sample event. There may be multiple analysis events for each
  sample event.
\end{itemize}

\begin{longtable}[c]{@{}lll@{}}
\toprule\addlinespace
Field & Data Type & Description
\\\addlinespace
\midrule\endhead
\textbf{sampleID} & AutoNumber & Primary Key for this table. Unique ID
for the sample event
\\\addlinespace
\textbf{stationID} & Short Text & Lookup primary Key from tblStation:
where was the sample taken?
\\\addlinespace
\textbf{sampleDate} & Short Date & Text Box: Date the sample was taken
in format MM/DD/YYYY
\\\addlinespace
\textbf{sampleTime} & Medium & Text Box: Time the sample was taken in
format HH:MM AM/PM
\\\addlinespace
\textbf{organization} & Short Text & Combo Box (``CRWA''; ``CTDEEP'';
``MEDEP''; ``NHDES'';``RIWW''; ``UNH\_CFB''; ``VTDEC''): Name of the
organization in charge of visit; if more than one choose the primary
\\\addlinespace
\textbf{fieldCrew} & Short Text & Text Box: Names of the field crew
separated by commas
\\\addlinespace
\textbf{sampleMethod} & Short Text & Combo Box (``Integrated Sampler''):
should be Integrated Sampler but other values can be added.
\\\addlinespace
\textbf{sampleDepthM} & Integer & Combo Box (1; 3): Depth (meters)
sample was taken. Should be 1 or 3 meters but other values can be added.
\\\addlinespace
\textbf{waterTempC} & Single & Text Box: Lake water temperature in
Celsius
\\\addlinespace
\textbf{weather} & Short Text & List Box (``Clear''; ``Partly Cloudy'';
``Overcast''; ``Rain''): Limited choice descriptor of weather conditions
\\\addlinespace
\textbf{surfaceWaterCondition} & Short Text & List Box (``Calm'';
``Ripples''; ``Choppy''; ``White Caps''): Limited choice descriptor of
weather conditions
\\\addlinespace
\textbf{photoSample} & Yes/No & Check Box: where photos taken during
sampling?
\\\addlinespace
\textbf{commentSample} & Long Text & Text Box: Additional information or
comments
\\\addlinespace
\bottomrule
\end{longtable}

\begin{itemize}
\itemsep1pt\parskip0pt\parsep0pt
\item
  \textbf{tblAnalysis} for each sample taken there will be one or more
  analysis events. This table provides general information on each
  analysis event. There may be multiple fluorometry readings for each
  analysis event.
\end{itemize}

\begin{longtable}[c]{@{}lll@{}}
\toprule\addlinespace
Field & Data Type & Description
\\\addlinespace
\midrule\endhead
\textbf{analysisID} & AutoNumber & Primary Key for this table. Unique ID
for the analysis event
\\\addlinespace
\textbf{sampleID} & AutoNumber & Lookup primary Key from tblSample:
analysis for which sample event?
\\\addlinespace
\textbf{analysisDate} & Short Date & Text Box: Date the sample was
analyzed in format MM/DD/YYYY
\\\addlinespace
\textbf{analysisTime} & Medium & Text Box: Time the sample was analyzed
in format HH:MM AM/PM
\\\addlinespace
\textbf{analystName} & Short Text & Text Box: Name of primary person in
charge of the analysis
\\\addlinespace
\textbf{frozen} & Yes/No & Check Box: was sample frozen prior to
analysis?
\\\addlinespace
\textbf{filtered} & Yes/No & Check Box: was sample filtered prior to
analysis?
\\\addlinespace
\textbf{dilution} & Short Text & Combo Box (``1:1''; ``1:5''; ``1:10''):
Default = 1 to 1 (not diluted); other choices are 1 to 5 and 1 to 10;
other values can also be added.
\\\addlinespace
\textbf{sampleTemperatureC} & Single & Text Box: sample temperature in
Celsius at time of analysis
\\\addlinespace
\textbf{photoAnalysis} & Yes/No & Check Box: where photos taken during
analysis?
\\\addlinespace
\textbf{commentAnalysis} & Long Text & Text Box: Additional information
or comments
\\\addlinespace
\bottomrule
\end{longtable}

\begin{itemize}
\itemsep1pt\parskip0pt\parsep0pt
\item
  \textbf{tblFluorometry} this table provides the fluorometry readings
  for each analysis. There should be at least one reading each for
  phycocyanin and chlorophyll a.
\end{itemize}

\begin{longtable}[c]{@{}lll@{}}
\toprule\addlinespace
Field & Data Type & Description
\\\addlinespace
\midrule\endhead
\textbf{fluorometryID} & AutoNumber & Primary Key for this table. Unique
ID for the fluorometry reading
\\\addlinespace
\textbf{analysisID} & AutoNumber & Lookup primary Key from tblAnalysis:
fluorometry reading for which analysis event?
\\\addlinespace
\textbf{parameter} & Short Text & List Box/Radio Button (``nearShore'';
``offShore'';``other'')
\\\addlinespace
\textbf{reading} & Single & Text Box: Fluorometry reading
\\\addlinespace
\textbf{units} & Short Text & List Box/Radio Button (``ug/l''; ``RFU''):
units of the fluorometry reading.
\\\addlinespace
\textbf{rep} & Integer & replicate number for multiple readings from a
sample
\\\addlinespace
\textbf{fluorometerType} & Combo Box: (``Beagle''): this should be a
Beagle but user can input other choices.
\\\addlinespace
\textbf{commentFluorometry} & Long Text & Text Box: Additional
information or comments
\\\addlinespace
\bottomrule
\end{longtable}

\end{document}
